\chapter{Introduction}
\label{ch:intro}
Nowadays, robots are widely used in various sectors such as manufacturing,
transport, earth and space exploration, laboratory, research, safety, etc..
Commercial and industrial robots are widespread today and used to perform jobs
more cheaply, more accurately and reliably than humans. They are also
employed in some jobs which are too dirty, dangerous and difficult for humans.
Even though land robots are of much use they have limitations. In case of
aerial surveillance, tracking, transport, photography Unmanned Aerial Vehicles
(UAV) more familiarly known as drones are of great use.
\section{Background and Motivation}
In India, drones have been reportedly used in search and rescue operation in
flood hit areas. Drones have been used also for military purposes e.g.,
surveillance to restrict infiltration in border areas. Still there are lot
of opportunities in various areas such as agriculture(for pest control),
public health (water puddle detection, stacked garbage detection), inspection
of large structure (dams, nuclear reactors, old temple restoration) etc..

There are two main challenges in using drones for such applications. One
of them is proper navigation and control of the drone, and another is analysis
of images captured from the camera to understand the scene. Developed countries
like US have been using military grade drones with superior technology and point
precision in different fields. But as these drones are very expensive, their use
is limited in developing countries. Also, as these drones use GPS for navigation
we cannot use them in indoor scenes. Even if GPS signal is available, we cannot
rely only on GPS for navigation of drones due to various reasons (GPS jamming,
spurious GPS signals). We may think of navigating drone manually in such cases.
But, it will be cumbersome and prone to human errors. Hence, there is need of
developing a stable technique for autonomous navigation and control of drones
in various scenarios.

Manual inspection of captured videos is very time consuming monotonic process
which makes it ineffective. So we need to develop an efficient algorithm to
process the video and output desirable representation of the input scene. For
example, if we are inspecting dam for cracks, we need to ensure that all small
cracks are detected along with their respective positions. This requires full
panoramic view of the dam constructed from closeups of the patches. But large
featureless regions of surfaces like dam challenges use of homography based
stitching to create the panorama. So there is requirement of a method which can
incorporate additional information available from drone to do mosaicing of
scenes with vacant spaces.


\section{Problem Statement}
  
\section{Contributions}
\subsection{Mosaicing Scenes with Vacant Spaces using Quadcopter}
\subsection{Autonomous Imaging of Multiplanar Regions through Quadcopter}
\subsection{A Motion Blur Resilient Fiducial For Quadcopter Imaging}
\subsection{Stagnant Water Detection}
\section{Organization of the thesis}
