\chapter{Introduction}
\label{ch:intro}
Nowadays, robots are widely used in various sectors such as manufacturing,
transport, earth and space exploration, laboratory, research, safety, etc..
Commercial and industrial robots are widespread today and used to perform jobs
more cheaply, more accurately and reliably than humans. They are also
employed in some jobs which are too dirty, dangerous and difficult for humans.
Even though land robots are of much use they have limitations. In case of
aerial surveillance, tracking, transport, photography Unmanned Aerial Vehicles
(UAV) more familiarly known as drones are of great use.
\section{Background and Motivation}
In India, drones have been reportedly used in search and rescue operation in
flood hit areas. Drones have been also used for military purposes e.g.,
surveillance to restrict infiltration in border areas. Still there are lot
of opportunities in various areas such as agriculture(for pest control),
public health (water puddle detection, stacked garbage detection), inspection
of large structure (dams, nuclear reactors, old temple restoration) etc..

There are two main challenges in such applications. First, proper navigation
and control of the drone, and second, analysis of images captured from the
camera to understand the scene. 
\section{Problem Statement}
The objective of the thesis is to
\section{Contributions}
\subsection{Mosaicing Scenes with Vacant Spaces using Quadcopter}
\subsection{Autonomous Imaging of Multiplanar Regions through Quadcopter}
\subsection{A Motion Blur Resilient Fiducial For Quadcopter Imaging}
\subsection{Stagnant Water Detection}
