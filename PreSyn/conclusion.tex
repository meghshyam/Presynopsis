\chapter{Conclusions and Future Work}
\label{sec:conclusion}
Unmanned Aerial Vehicles (UAVs) such as quadcopters have made it possible to
image human inaccessible areas for various applications such as inspection of
dams, art galleries. Manual navigation of the quadcopter is a very tedious task
which gives inaccurate results. Hence there is need of a robust technique  for
accurate navigation of quadcopter for imaging surfaces. We also need
to need to construct a suitable representation of an input scene such that all
details are precisely captured. 

In this thesis, we have presented a method for autonomous navigation of
quadcopter for imaging scene spread over the multiplanar surface. This method
first estimates the path along which quadcopter needs to be maneuvered. It also
finds out the optimal positions from which we need to capture the images so that
those images encompass the whole scene. Finally, we accurately maneuever the
quadcopter along the estimated path and output unrolled view of the scene spread
over multiplanar surface.

Many times surfaces such as dams, art galleries contain large featureless
regions. State of the art stitching algorithms such as Adobe Photoshop fails to
create complete panorama due to the failure of feature matching algorithm to
find enough matches. We have developed an algorithm for creation of mosaic in such cases to
get the complete panorama. This algorithm leverages the positional information
available from calibrated quadcopter to solve ``vacant'' spaces problem.

We have also focussed on the design of blur resilient fiducials for tracking of
objects. Fiducial is designed in such a way that the embedded code remains
intact irrespective of the direction of blur. We have developed a fiducial
detection method based on Principal Component Analysis (PCA) of Gabor filter
output on the input scene. We can put these fiducials on the quadcopter so that
it can be identified uniquely in the case of collaboration among multiple
quadcopters.

\section{Future Work}
Our method for autonomous navigation of quadcopter to image multiplanar
surfaces works only if all surfaces are seen from a single point. However,
there may be cases where all surfaces cannot be seen from a single point. One
can extend our work so that in first flight we will collect information of all
surfaces, plan the estimated path and then maneuver the quadcopter autonomously
along the estimated path. Another extension of this work also can be done in
direction of handling of any parametric surface compared to just planar
surfaces.

Currently, our fiducial detection algorithm is not real-time. This can be
improved by using parallel implementation of Gabor filter using GPU-based
system. The number of fiducials can also be increased by using color-based
codes.