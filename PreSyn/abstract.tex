\chapter*{Abstract}
The necessity for flying device with greater maneuverability and hovering
ability led to the creation of a quadcopter.  Due to their ability to go to
inaccessible areas e.g., terraces of high-rise buildings, hills, etc. and
capturing high-quality images from the onboard camera, it is indispensable in
different applications (e.g., search and rescue,  surveillance etc.).
Areas such as agriculture(for pest control), public health (stacked garbage
detection), civil inspection (dams, nuclear reactors, old temple restoration)
have opened a lot of opportunities towards the utility of quadcopters.

Applications such as inspection of towers as well as art galleries require
autonomous imaging of multiplanar scene to provide the unrolled view of the
scene. We cannot image large multiplanar surfaces with the help of single
quadcopter due to battery constraints. Here, the necessity of using multiple
quadcopters arises. The collaboration between these quadcopters can be achieved
by tracking each of them. Fiducials come to our rescue for tracking objects in
the environment. However, jerky nature of inexpensive quadcopters introduces
motion blur in the captured image which poses a problem in detecting existing
fiducials. Hence, motion blur resilient fiducials are needed for robust
detection of objects.

We would like to get an orthographic panoramic view of a large
multiplanar surface for precise details. Manual navigation of quadcopter for
imaging large planar surfaces is very tedious. It is also impossible to
maintain quadcopter's camera normal to the imaging plane. Hence there is need
of a technique for autonomous navigation of the quadcopter to image
multiplanar scene in an efficient manner.

Mosaicing of multiplanar scenes to get unrolled view pose a challenge to
homography based stitching methods. Mosaicing of scenes containing vacant
spaces or featureless regions on a single planar surface is also not possible
with present state of the art stitching methods. We need to leverage the
additional information from quadcopter to overcome these problems and create
a complete panorama of input scene.

Our work focuses on the above challenges for the imaging of multiplanar scenes
through quadcopter. We have developed an algorithm which solves the ``vacant
spaces'' problem in mosaicing of a planar scene by using positional information.
We have also innovated a vision based technique for autonomous navigation of
quadcopter for efficiently imaging scene spread over a multiplanar surface. We
capture each plane from normal viewpoints and output an unrolled view of the
scene. Finally, we have designed a blur resilient fiducial for tracking of
the quadcopter.