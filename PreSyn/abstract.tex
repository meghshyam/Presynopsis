\chapter*{Abstract}
\thispagestyle{empty}
Digital imaging is being extensively used in almost all sectors in today's
world. However, it is quite difficult to take good pictures from a handheld
camera in some situations, e.g., inspection of nuclear reactors, dams etc. Recently, Unmanned
Aerial Vehicles (UAV) have gained focus for imaging in such situations. A
quadcopter, or a quadrotor helicopter, is one of the unpiloted aerial vehicles
that has great maneuverability and hovering ability. With its small size and
agile maneuverability, quadcopter can be flown indoors as well as
outdoors. Due to its ability to go to inaccessible areas e.g., terraces of
high-rise buildings, hills, etc. and capturing high-quality images from the
onboard camera, it has become popular in various applications such as search and
rescue.

Applications such as inspection of dam require a high level of details in the
final output. Hence, we need to image such planar objects from close range in
the normal direction. Quadcopters can be flown from close range to capture such
details. However, the manual inspection of the captured video is time consuming
as well as prone to error. We have to create a suitable representation from the
captured video so that even minute details (e.g., small cracks in the dam) are
detected accurately. One such representation can be panoramic construction from
the video encompassing the whole scene. 

There are two problems in creating a mosaic from the captured video. First,
a number of images from captured video are beyond the capacity of standard
mosaicing techniques such as Adobe Photoshop. Second, if there are spaces with very less
or no ``features'' (e.g., textureless regions on a wall of a dam) or patterns
repeated in a scene (e.g., modern art posters in an art exhibition), it is not possible for existing
mosaicing techniques to construct a panorama accurately. The reason behind 
the failure is, standard mosaicing techniques fully rely upon the matching
algorithms to find appropriate matches between images. 

We have solved these two problems by leveraging information from the quadcopter.
Our helper to solve these problems is, Inertial Measurement Unit (IMU),
which is present on any quadcopter. The IMU can give us a reasonable estimate of
the position from where each image is captured. We have developed an image selection
algorithm which uses the positional data from calibrated IMU to discover optimal
images encompassing the input scene. Standard stitchers match each image
with every other image for finding suitable pairs before stitching them
together. However, as we arrange selected images into rectangular grid
according to their position, we just need to match images which are present in
the 8-neighborhood of a given image.

If there are vacant spaces in an input scene we will get multiple mini-panoramas
representing disconnected parts of the scene as a mosaicing output.
The mini-panoramas are then joined together appropriately with the help of
positional information to form the super-panorama. The efficacy of the approach
is demonstrated on a number of input sequences that cannot be mosaiced by existing methods.

Though we are successful in mosaicing an input scene spread over a single planar
surface, in the real world, we will encounter mainly multiplanar and curved
surfaces. Manually navigating the quadcopter around such surfaces is very
tedious as well as not effective. We have presented a method for autonomous
navigation of quadcopter for imaging large multiplanar surfaces. The method
uses Parallel Tracking and Mapping (PTAM) based approach to create an 
approximate sparse 3D map of the input scene which will be used to fit
multiplanar bounded regions. Later, the positions to image each bounded region
(with unique normal) independently are determined and quadcopter is
autonomously maneuvered along those positions to image the area specified by
the user. The goal is to create the ``unrolled'' view of the input scene where we
get the output mosaic of the input scene as if it is present on a single plane.
Hence, we have developed an algorithm which first creates a mosaic of each
planar region using homography based stitching and later joins the mosaics of
each planar region together using plane information and camera positions. We
have shown the potency of the method on datasets imaged on various combinations
of multiplanar surfaces.

A quadcopter's limited battery is a major hurdle we experienced during
the imaging of multiplanar scene. Typical quadcopter's battery sustains only for
20-30 minutes which is not enough for imaging large scenes in single flight. We can use
multiple quadcopters for imaging large multiplanar surface to overcome the
battery issues. However, we have to maintain collaboration among these
quadcopters for efficient capture. Quadcopters have to keep track of each other
in order to divide the work in an appropriate manner. Fiducials come to our
rescue for tracking objects in the environment. Applying fiducials on quadcopters can
help in collaboration between them to image large multiplanar surfaces.
Quadcopters are subject to quick and unstable motions that can cause
significant motion blur in the captured images. This severely affects the
detection rate of existing fiducials. 

We propose the design of a fiducial that is resilient to motion blur. The
design of contrasting concentric rings is based on the observation that the
direction perpendicular to the motion blur direction will be unaffected by the
blur and therefore still be recognizable. It is shown through experimental
validation that our fiducial will work under large amounts of motion blur and
can significantly outperform existing fiducials under this scenario. It is also
demonstrated that using such fiducials one can tell from which side of the
quadcopter we are gazing.

Our overall work focuses on overcoming the challenges faced during imaging of
the multiplanar scenes through quadcopter. We have developed an algorithm which
solves the ``vacant spaces'' problem in mosaicing of a planar scene by using
positional information (IMU). We have also proposed a vision based technique for
autonomous navigation of quadcopter for efficiently imaging scene spread over a
multiplanar surface. Each plane is imaged from normal viewpoints and output an
unrolled panoramic view of the scene. Finally, we have designed a blur resilient
fiducial for tracking of the quadcopter in the environment.        